\chapter*{Introduction} % chapter* je necislovana kapitola
\addcontentsline{toc}{chapter}{Introduction} % rucne pridanie do obsahu
\markboth{Introduction}{Introduction} % vyriesenie hlaviciek

One of the ways that make nanopore DNA sequencing more cost-effective is the utilization of barcoding. Barcoding (or barcode multiplexing) is a technology that enables simultaneous sequencing of several DNA samples in a single run of the sequencer. The idea behind barcoding is that the fragments from a particular sequenced sample are uniquely labeled by a short DNA sequences called barcodes. The inverse process to barcoding is the debarcoding (or demultiplexing) of the barcoded DNA reads by assigning them a label by the barcode they contain, and therefore identifying their origin.

As of now, several algorithms have been designed for the barcode demultiplexing problem, most of them operating with base called DNA sequences, i.e. sequences consisting of symbols A, C, G, T. Base calling is, however, erroneous process and when errors occur in the part of the sequence that corresponds to a barcode, the read may end up unidentifiable. These algorithms (Albacore, Porechop) typically render $\approx 20\%$ of the reads unusable \cite{Deepbinner}.

Recent approach of Wick et al. \cite{Deepbinner} used a convolutional neural network (CNN) to classify reads by their raw sequencing signal and achieved a state-of-the-art accuracy of $\approx 98\%$ of correctly assigned reads, with only $\approx 5\%$ of reads lost. Their CNN, however, requires to be trained on a large dataset of signals that contain the corresponding barcodes.

In this thesis we present an unsupervised approach to barcode demultiplexing. Our algorithm operates with raw signals as Deepbinner, but does not require to be trained on particular barcodes. More specifically, our method does not assume the knowledge of the structure of barcode sequences, hence can be used with any available barcode sequencing kit.

In Chapter \ref{kap:sequencing} we present an introduction into DNA sequencing, the usage of barcoding and give an overview of existing barcode demultiplexing tools. In Chapter \ref{kap:outline} we give a high-level overview of the unsupervised barcode demultiplexing algorithm that we propose, with an identification of its main features and problems that we will address later in this thesis. Chapter \ref{kap:align} gives an overview of known algorithms and methodologies for alignment of biological sequences and explains the ideas of the similarity function that we will later use to compare barcoded reads. Chapter \ref{kap:data} presents information about the datasets we will be working with, the methods used for their preprocessing and a set of experiments that were performed. Finally, Chapter \ref{kap:clustering} deals with the problem of clustering of reads according to their barcode classes and summarizes the performance of our methods in a number of experiments.